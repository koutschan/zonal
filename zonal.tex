\documentclass[10pt,oneside,american]{amsart}
\usepackage[T1]{fontenc}
\usepackage[latin9]{inputenc}
\setcounter{secnumdepth}{2}
\setcounter{tocdepth}{2}
\usepackage{amstext}
\usepackage{amsthm}
\usepackage{amssymb}
\usepackage{graphicx}

\makeatletter
%%%%%%%%%%%%%%%%%%%%%%%%%%%%%% Textclass specific LaTeX commands.
\numberwithin{equation}{section}
\numberwithin{figure}{section}
  \theoremstyle{definition}
  \newtheorem{defn}{\protect\definitionname}[section]
  \theoremstyle{remark}
  \newtheorem{rem}{\protect\remarkname}[section]
  \theoremstyle{plain}
  \newtheorem{prop}{\protect\propositionname}[section]
  \theoremstyle{definition}
  \newtheorem{example}{\protect\examplename}[section]
  \theoremstyle{plain}
  \newtheorem{thm}{\protect\theoremname}[section]
  \theoremstyle{plain}
  \newtheorem{lemma}{Lemma}[section]
  \theoremstyle{plain}
  \newtheorem{conj}{Conjecture}[section]

%%%%%%%%%%%%%%%%%%%%%%%%%%%%%% User specified LaTeX commands.
\DeclareMathOperator{\sech}{sech}
\DeclareMathOperator{\diag}{diag}
\DeclareMathOperator{\Tr}{tr}
\newcommand{\SPD}{\operatorname{SPD}}

\makeatother

\usepackage{babel}
\providecommand{\definitionname}{Definition}
\providecommand{\examplename}{Example}
\providecommand{\propositionname}{Proposition}
\providecommand{\remarkname}{Remark}
\providecommand{\theoremname}{Theorem}

\begin{document}

\title{Definition and Calculation of Zonal Polynomials}

\author{Lin Jiu and Christoph Koutschan}
\begin{abstract}
In this report, we summarize different definitions of zonal polynomials
involving statistics, differential geometry, representation theory
and combinatorics. A manual of a package for calculating zonal polynomials
is also introduced.
\end{abstract}

\maketitle

\section{Introduction}

In the project proposal \cite[pp.~8]{Proposal}, G.(goal)~2 aims
at hypergeometric function with matrix argument, defined as follows.
\begin{defn}
Given an $m\times m$ symmetric, positive-definite matrix $Y$, the
hypergeometric function with matrix argument is defined as 
\begin{equation}
_{p}F_{q}\left(\genfrac{}{}{0pt}{}{a_{1},\ldots,a_{p}}{b_{1},\ldots,b_{q}}\bigg|Y\right):=\sum_{n=0}^{\infty}\sum_{\lambda\in\mathcal{P}_{n}}\frac{\left(a_{1}\right)_{\lambda}\cdots\left(a_{p}\right)_{\lambda}}{\left(b_{1}\right)_{\lambda}\cdots\left(b_{q}\right)_{\lambda}}\cdot\frac{\mathcal{C}_{\lambda}\left(Y\right)}{n!},\label{eq:MatrixpFq}
\end{equation}
where 
\begin{itemize}
\item $\mathcal{P}_{n}$ is the set of all integer partitions of $n$, where
every partition $\lambda\in\mathcal{P}_{n}$ is defined to be a tuple
$\lambda=\left(\lambda_{1},\ldots,\lambda_{k}\right)$ such that 
\[
\lambda_{1}\geq\lambda_{2}\geq\cdots\geq\lambda_{k}>0
\]
and 
\[
\lambda_{1}+\cdots+\lambda_{k}=n;
\]
\item the Pochhammer symbol, for positive integer $m$, $\left(a\right)_{m}:=a\left(a+1\right)\cdots\left(a+m-1\right)$
is applied to define
\[
\left(a\right)_{\lambda}=\left(a\right)_{\left(\lambda_{1},\dots,\lambda_{k}\right)}:=\prod_{i=1}^{k}\left(a-\frac{i-1}{2}\right)_{\lambda_{i}};
\]
\item and finally $\mathcal{C}_{\lambda}(Y)$ denotes the zonal polynomial
of $Y$, indexed by a partition $\lambda$, which is a symmetric polynomial
of degree $n$, in the eigenvalues $y_{1},\ldots,y_{m}$ of $Y$,
satisfying 
\begin{equation}
\sum_{\lambda\in\mathcal{P}_{n}}\mathcal{C}_{\lambda}\left(Y\right)=\left(\Tr Y\right)^{n}=\left(y_{1}+\cdots+y_{m}\right)^{n}.\label{eq:TrZonal}
\end{equation}
\end{itemize}
\end{defn}
Hashiguchi et.~al \cite{1F1} study the case $p=q=1$, using the holonomic
gradient method \cite[pp.~4--6]{Proposal}. Comparing to the usual
hypergeometric function
\[
_{p}F_{q}\left(\genfrac{}{}{0pt}{}{a_{1},\ldots,a_{p}}{b_{1},\ldots,b_{q}}\bigg|z\right):=\sum_{n=0}^{\infty}\frac{\left(a_{1}\right)_{n}\cdots\left(a_{p}\right)_{n}}{\left(b_{1}\right)_{n}\cdots\left(b_{q}\right)_{n}}\cdot\frac{z^{n}}{n!},
\]
we see that the key is the zonal polynomial $\mathcal{C}_{\lambda}(Y)$,
which inspires this report. Definitions will be presented in Section
\ref{sec:DEFS} and a short manual on a \texttt{Sage} package we implemented
for calculation of $\mathcal{C}_{\lambda}(Y)$ follows
in Section \ref{sec:Package}.

\section{\label{sec:DEFS}Definitions of zonal polynomials}

We shall summarize four different definitions of zonal polynomials
involving statistics, differential geometry, representation theory
and combinatorics. Namely each subsection will present one aspect. 

First of all, we need an important linear space.
\begin{defn}
Let $V_{n}$ be the space of symmetric homogeneous polynomials of
degree $n$ in the variables $y_{1},\ldots,y_{m}$ including the zero polynomial.
Namely, if $f\in V_{n}$, we have 
\begin{itemize}
\item $\deg f=n$ or $f\equiv0$;
\item if $\deg f=n$, then $f$ is symmetric and homogeneous in $y_{1},\ldots,y_{m}$.
\end{itemize}
Moreover, any polynomial $f\in V_{n}$ can also be viewed as a polynomial
in the eigenvalues of $m\times m$ symmetric, positive-definite matrices.
Denote the space of $m\times m$ symmetric, positive-definite matrices
as $\SPD(m)$. 
\end{defn}

\subsection{Definition involving the Wishart distribution}
\begin{defn}
Define the elementary symmetric polynomial \cite[pp.~12, eq.~9]{Takemura}
\[
u_{r}(x_{1},\ldots,x_{m}):=\underset{i_{1}<\cdots<i_{r}}{\sum}x_{i_{1}}\cdots x_{i_{r}}.
\]
Then, for $\lambda=\left(\lambda_{1},\ldots,\lambda_{l}\right)\in\mathcal{P}_{n}$,
the polynomials \cite[pp.~12, eq.~10]{Takemura}
\[
\mathcal{U}_{\lambda}:=u_{1}^{\lambda_{1}-\lambda_{2}}u_{2}^{\lambda_{2}-\lambda_{3}}\cdots u_{l-1}^{\lambda_{l-1}-\lambda_{l}}u_{l}^{\lambda_{l}}
\]
\cite[pp.~13]{Takemura} form a basis for $V_{n}$.
\end{defn}
\begin{rem}
Obviously, 
\[
\deg\mathcal{U}_{\lambda}=\left(\lambda_{1}-\lambda_{2}\right)+2\left(\lambda_{2}-\lambda_{3}\right)+\cdots+l\lambda_{l}=\lambda_{1}+\cdots+\lambda_{l}=n.
\]
Further associate a lexicographical order to $\mathcal{P}_{n}$ \cite[pp.~9, eq.~7]{Takemura}:
for $p=\left(p_{1},\ldots,p_{j}\right),q=\left(q_{1},\ldots,q_{l}\right)\in\mathcal{P}_{n}$,
\[
p>q\Leftrightarrow p_{1}=q_{1},\cdots p_{k-1}=q_{k-1},p_{k}>q_{k}\text{ for some }k.
\]
Then, we could write the basis formed by the $\mathcal{U}_{\lambda}$
as a column vector:
\[
\mathcal{U}:=\left(\mathcal{U}_{\left(n\right)},\mathcal{U}_{\left(n-1,1\right)},\ldots,\mathcal{U}_{\left(1,\ldots,1\right)}\right)^{T}.
\]
\end{rem}
Since this section presents a definition of $\mathcal{C}_{\lambda}(Y)$
related to the Wishart distribution, we shall first define it, as follows.
\begin{defn}
Let $X_{\nu\times m}$ be such that each row is independently drawn
from an $m$-variate normal distribution of mean $0$ and covariance
matrix $V$, namely, 
\[
  \left(x_{i}^{1},\ldots,x_{i}^{m}\right)\sim\mathcal{N}_{m}(0,V)
  \qquad (1\leq i\leq\nu).
\]
Then, we say $S:=X^{T}X$ has the Wishart distribution, denoted by
\[
S=X^{T}X\sim W_{m}(V,\nu),
\]
where $\nu$ is called the degree of freedom.
\end{defn}
\begin{rem}
Recall that in the $1$-dimensional case if $Z_{1},\ldots,Z_{k}\sim\mathcal{N}(0,1)$
are independent Gaussian distributed, then $Q:=Z_{1}+\cdots+Z_{k}\sim\chi_{k}^{2}$.
In other words, the sum of independent Gaussian distributions is chi-square
distributed. The Wishart distribution can be viewed as a multi-dimensional
generalization of the chi-square distribution.
\end{rem}
Define the linear transform \cite[Lem.~2, pp.~19]{Takemura} $\tau_{\nu}\colon V_{n}\longrightarrow V_{n}$,
for $Y\in \SPD(m)$, 
\[
  \left(\tau_{\nu}(\mathcal{U}_{\lambda})\right)\left(Y\right) :=
  \mathbb{E}_{W}\left[\mathcal{U}_{\lambda}(YW)\right]\text{ for }W\sim W(I_m,\nu).
\]
As $\mathcal{U}$ forms a basis of $V_{n}$, $\tau_{\nu}(\mathcal{U}):=\left(\tau_{\nu}(\mathcal{U}_{\left(n\right)}),\tau_{\nu}(\mathcal{U}_{\left(n-1,1\right)}),\ldots,\tau_{\nu}(\mathcal{U}_{\left(1,\ldots,1\right)})\right)^{T}$
must be a linear combination of $\mathcal{U}$ \cite[eq.~19, pp.~22]{Takemura},
denoted by 
\[
\tau_{\nu}(\mathcal{U})=T_{\nu}\mathcal{U}.
\]
Properties of $T_{\nu}$ guarantee that \cite[Lem.~4, pp.~22]{Takemura}
\[
T_{\nu}=\Xi^{-1}\Lambda_{\nu}\Xi,
\]
for some nonsingular upper triangular matrix $\Xi$, which is uniquely
determined up to a (possibly different) multiplicative constant for
each row, and 
\[
\Lambda_{\nu}=\diag\left(2^{n}\overset{k}{\underset{i=1}{\prod}}\left(\frac{\nu+1-i}{2}\right)_{\lambda_{i}},\lambda=\left(\lambda_{1},\ldots,\lambda_{k}\right)\in\mathcal{P}_{n}\right).
\]
Now, we can define zonal polynomials.
\begin{defn}
The zonal polynomial $\mathcal{Y}_{\lambda}$, for a partition $\lambda\in\mathcal{P}_{n}$,
is defined by a vector form
\[
\mathcal{Y}=\left(\begin{matrix}\mathcal{Y}_{\left(n\right)}\\
\mathcal{Y}_{\left(n-1,1\right)}\\
\vdots\\
\mathcal{Y}_{\left(1,\ldots,1\right)}
\end{matrix}\right)=\Xi\,\mathcal{U}=\Xi\left(\begin{matrix}\mathcal{U}_{\left(n\right)}\\
\mathcal{U}_{\left(n-1,1\right)}\\
\vdots\\
\mathcal{U}_{\left(1,\ldots,1\right)}
\end{matrix}\right),
\]
and $\mathcal{C}_{\lambda}(Y)=d_{\lambda}\mathcal{Y}_{\lambda}(Y)$,
where the constants $d_\lambda$ are given by
\[
  d_{\lambda}=\frac{\underset{i<j}{\prod}\left(2\lambda_{i}-2\lambda_{j}-i+j\right)}{\overset{k}{\underset{i=1}{\prod}}\left(2\lambda_{i}+k-i\right)!}
  \cdot\frac{2^{n}n!}{\left(2n\right)!}.
\]
\end{defn}

\subsection{Definition in differential geometry}

We first recall the Laplace-Beltrami operator on Riemannian manifolds.
\begin{defn}
On a Riemannian manifold $\left(M,g\right)$, the Laplace-Beltrami
operator on $f\in C^{\infty}(M)$ is given by 
\[
\Delta f:=\left(\mathrm{div}\bullet\mathrm{grad}\right)f=\sum_{i=1}^{n}\frac{1}{\sqrt{G}}\partial_{k}\left(g^{ik}\sqrt{G}\partial_{i}f\right),
\]
where $n=\dim M$, $\left(g_{ij}\right)_{n\times n}$ is the metric
matrix and $G:=\det\left(g_{ij}\right)$. 
\end{defn}
\begin{rem}
When $M=\mathbb{R}^{n}$ and $\left(g_{ij}\right)=I_{n}$, the identity
matrix, then 
\[
\Delta f=\sum_{i=1}^{n}\frac{\partial^{2}f}{\partial x_{i}^{2}}.
\]
Namely, the Laplace-Beltrami operator is the generalization of the Laplace
operator on $\mathbb{R}^{n}$. 
\end{rem}
\begin{prop}
Given $X\in \SPD(m)$, James \cite[eq.~3.12, pp.~1712]{James1} derived
that the Laplace-Beltrami operator is given by
\[
\Delta=\sum_{i=1}^{m}\left(y_{i}^{2}\frac{\partial^{2}}{\partial y_{i}^{2}}-\frac{m-3}{2}y_{i}\frac{\partial}{\partial y_{i}}+\sum_{j=1,j\neq i}^{m}\frac{y_{i}^{2}}{y_{i}-y_{j}}\frac{\partial}{\partial y_{i}}\right),
\]
where $X=HYH^{T}$, for $H\in O(n)$ and $Y=\diag(y_{1},\ldots,y_{n})$. 
\end{prop}
\begin{rem}
The second term in the parentheses is, up to the constant $\frac{m-3}{2}$,
the Euler's operator $\overset{m}{\underset{i=1}{\sum}}y_{i}\frac{\partial}{\partial y_{i}}$,
which has all symmetric, homogeneous polynomials as its eigenfunctions.
Thus, when considering any eigenfunction of the Laplace-Beltrami operator~$\Delta$,
this term can be eliminated.
\end{rem}
\begin{defn}
The zonal polynomials $\mathcal{C}_{\lambda}(y_{1},\ldots,y_{m})$
are eigenfunctions of $\Delta_{Y}$, defined by
\[
  \Delta_{Y}:=\sum_{i=1}^{m}\left(y_{i}^{2}\frac{\partial^{2}}{\partial y_{i}^{2}}+\sum_{j=1,j\neq i}^{m}\frac{y_{i}^{2}}{y_{i}-y_{j}}\frac{\partial}{\partial y_{i}}\right).
\]
In particular
\[
  \Delta_{Y}\mathcal{C}_{\lambda}(Y)=\left(\rho_{\lambda}+(m-1)n\right)\mathcal{C}_{\lambda}(Y)
  \qquad (n=\lambda_1+\dots+\lambda_k),
\]
where
\begin{equation}
  \rho_{\lambda}:=\sum_{i=1}^{k}\lambda_{i}\left(\lambda_{i}-i\right).\label{eq:RHO}
\end{equation}
\end{defn}

\subsection{Definition through representation theory}

Consider the general linear group $G=\mathrm{GL}(m)$ on
$V_{n}$. Define a representation 
\begin{eqnarray*}
g\in\mathrm{GL}(m):V_{n} & \rightarrow & V_{n}\\
\varphi\left(Y\right) & \mapsto & \varphi\left(g^{-1}Y\left(g^{-1}\right)^{T}\right)
\end{eqnarray*}
As a representation, the linear space can be decomposed into invariant
subspaces \cite[pp.~611]{Representation}
\[
V_{n}=\bigoplus_{\lambda\in\mathcal{P}_{n}}V_{\lambda}.
\]

\begin{defn}
For given $Y\in \SPD(m)$ and a partition $\lambda\in\mathcal{P}_{n}$,
the projection
\begin{equation}
\left(\Tr Y\right)^{n}\bigg|_{V_{\lambda}}=\mathcal{C}_{\lambda}(Y)\label{eq:projection}
\end{equation}
defines the zonal polynomials.
\end{defn}
\begin{rem}
\eqref{eq:projection} confirms \eqref{eq:TrZonal}. 
\end{rem}

\subsection{A short remark on Macdonald, Jack and zonal polynomials}

Special case of Macdonald polynomial gives Jack polynomials $J_{\lambda}^{\left(\alpha\right)}$,
which when $\alpha=2$ gives zonal polynomials. 

\section{\label{sec:Package}Sage package for the calculation of zonal polynomials}

Although there are more than one ways to define the zonal polynomial $\mathcal{C}_{\lambda}(Y)$,
in practice, none of these definition gives an algorithm or formula
to directly compute $\mathcal{C}_{\lambda}(Y)$. Now, we
follow the steps by Muirhead \cite{Muirhead} to build up a package for the
calculation of $\mathcal{C}_{\lambda}(Y)$.
\begin{defn}
For $\lambda=\left(\lambda_{1},\ldots,\lambda_{l}\right)\in\mathcal{P}_{n}$,
define the \emph{monomial symmetric function} as 
\begin{equation}
\mathcal{M}_{\lambda}(y_{1},\ldots,y_{m})=\sum_{\genfrac{}{}{0pt}{}{i_{1},\ldots,i_{l}}{\text{distinct terms}}}y_{i_{1}}^{\lambda_{1}}\cdots y_{i_{l}}^{\lambda_{l}}=y_{1}^{\lambda_{1}}\cdots y_{l}^{\lambda_{l}}+\text{symmetric terms}.\label{eq:MZonal}
\end{equation}
\end{defn}
\begin{example}
Let $Y$ be an $m\times m$ symmetric matrix with eigenvalues $y_{1},\ldots,y_{m}$:
\begin{enumerate}
\item 
\[
\mathcal{M}_{\left(1\right)}(Y)=y_{1}+\cdots+y_{m};
\]
\item 
\[
\mathcal{M}_{\left(2\right)}(Y)=y_{1}^{2}+\cdots+y_{m}^{2};
\]
\item \label{enu:ILessThanJ}
\[
\mathcal{M}_{\left(1,1\right)}(Y)=\underset{i<j}{\sum}y_{i}y_{j};
\]
\item \label{enu:IANDJ}
\[
\mathcal{M}_{\left(2,1\right)}(Y)=\underset{i,j}{\sum}y_{i}^{2}y_{j}.
\]
\end{enumerate}
Here we see the last two functions have the same length of partitions
but different sum ranges, due to the fact that $y_{1}y_{2}$ and $y_{2}y_{1}$
are the same while $y_{1}^{2}y_{2}$ and $y_{2}^{2}y_{1}$ are different. 
\end{example}
\begin{rem}
An explicit expression of $M_{\lambda}(Y)$ is given by
\cite[pp.~11, eq.~6]{Takemura}:
\begin{equation}
\mathcal{M}_{\left(1^{m_{1}}2^{m_{2}}\cdots\right)}\left(Y\right)=\left(\prod_{i=1}^{h}\frac{1}{m_{i}!}\right)\sum_{i_{1},\ldots,i_{l}}y_{i_{1}}^{\lambda_{1}}\cdots y_{i_{l}}^{\lambda_{l}}.\label{eq:MZonalComputation}
\end{equation}
\end{rem}
\begin{thm}
A general result by James \cite{James1} shows that
\begin{equation}
C_{\kappa}(Y)=\sum_{\lambda\leq\kappa}c_{\kappa,\lambda}M_{\lambda}(Y),\label{eq:CInTermsOfM}
\end{equation}
 for some constants $c_{\kappa,\lambda}$. (see also \cite[pp.~234, eq.~13]{Muirhead}).
\end{thm}
\begin{example}
\label{Tables}The following tables \cite[pp.~238]{Muirhead} show
constants $c_{\kappa,\lambda}$, in the case $n=\left|\kappa\right|=\left|\lambda\right|=2,\ldots5$. 
\begin{itemize}
\item $n=2$
\[
\def\arraystretch{1.3}
\begin{array}{c|cc}
\kappa\backslash\lambda & \left(2\right) & \left(1,1\right)\\ \hline
\left(2\right) & 1 & \frac{2}{3}\\
\left(1,1\right) & 0 & \frac{4}{3}
\end{array}
\]
\item $n=3$
\[
\def\arraystretch{1.3}
\begin{array}{c|ccc}
\kappa\backslash\lambda & \left(3\right) & \left(2,1\right) & \left(1,1,1\right)\\ \hline
\left(3\right) & 1 & \frac{3}{5} & \frac{2}{5}\\
\left(2,1\right) & 0 & \frac{12}{5} & \frac{18}{5}\\
\left(1,1,1\right) & 0 & 0 & 2
\end{array}
\]
\item $n=4$
\[
\def\arraystretch{1.3}
\begin{array}{c|ccccc}
\kappa\backslash\lambda & \left(4\right) & \left(3,1\right) & \left(2,2\right) & \left(2,1,1\right) & \left(1,1,1,1\right)\\ \hline
\left(4\right) & 1 & \frac{4}{7} & \frac{18}{35} & \frac{12}{35} & \frac{8}{35}\\
\left(3,1\right) & 0 & \frac{24}{7} & \frac{16}{7} & \frac{88}{21} & \frac{32}{7}\\
\left(2,2\right) & 0 & 0 & \frac{16}{5} & \frac{32}{15} & \frac{16}{5}\\
\left(2,1,1\right) & 0 & 0 & 0 & \frac{16}{3} & \frac{64}{5}\\
\left(1,1,1,1\right) & 0 & 0 & 0 & 0 & \frac{16}{5}
\end{array}
\]
\item $n=5$
\[
\def\arraystretch{1.3}
\begin{array}{c|ccccccc}
\kappa\backslash\lambda & \left(5\right) & \left(4,1\right) & \left(3,2\right) & \left(3,1,1\right) & \left(2,2,1\right) & \left(2,1,1,1\right) & \left(1,1,1,1,1\right)\\ \hline
\left(5\right) & 1 & \frac{5}{9} & \frac{10}{21} & \frac{20}{63} & \frac{2}{7} & \frac{4}{21} & \frac{8}{63}\\
\left(4,1\right) & 0 & \frac{40}{9} & \frac{8}{3} & \frac{46}{9} & 4 & \frac{14}{3} & \frac{40}{9}\\
\left(3,2\right) & 0 & 0 & \frac{48}{7} & \frac{32}{7} & \frac{176}{21} & \frac{64}{7} & \frac{80}{7}\\
\left(3,1,1\right) & 0 & 0 & 0 & 10 & \frac{20}{3} & \frac{130}{7} & \frac{200}{7}\\
\left(2,2,1\right) & 0 & 0 & 0 & 0 & \frac{32}{3} & 16 & 32\\
\left(2,1,1,1\right) & 0 & 0 & 0 & 0 & 0 & \frac{80}{7} & \frac{800}{21}\\
\left(1,1,1,1,1\right) & 0 & 0 & 0 & 0 & 0 & 0 & \frac{16}{3}
\end{array}
\]
\end{itemize}
\end{example}
\begin{thm}
The constant $c_{\kappa,\lambda}$ satisfies the recurrence \cite[pp.~234, eq.~14]{Muirhead}
\begin{equation}
c_{\kappa,\lambda}=\sum_{\lambda<\mu\leq\kappa}\frac{\left(\lambda_{i}+t\right)-\left(\lambda_{j}-t\right)}{\rho_{\kappa}-\rho_{\lambda}}c_{\kappa,\mu},\label{eq:CRec}
\end{equation}
where for $\lambda=\left(\lambda_{1},\ldots,\lambda_{l}\right)$,
the sum is over all $\mu=\left(\lambda_{1},\ldots,\lambda_{i}+t,\ldots,\lambda_{j}-t,\ldots,\lambda_{l}\right)$
for $t=1,\ldots,\lambda_{j}$ such that by rearranging the tuple~$\mu$
in a descending order, it lies as $\lambda<\mu\leq\kappa$.
\end{thm}
\begin{rem}
Observing \eqref{eq:CRec}, once given initial $c_{\kappa,\kappa}$,
$\forall\lambda<\kappa$, it could compute $c_{\kappa,\lambda}$.
Now, observing tables in Example \ref{Tables} and recall \eqref{eq:TrZonal},
it is easy to see that for each column, the sum is given by a multinomial
coefficient. More concretely, let $\lambda=\left(\lambda_{1},\ldots,\lambda_{l}\right)\in\mathcal{P}_{n}$,
\begin{equation}
  \sum_{\kappa=\lambda}^{\left(n\right)}c_{\kappa,\lambda}=\binom{n}{\lambda_{1},\ldots,\lambda_{l}}.\label{eq:CInitial}
\end{equation}
In particular, $c_{\left(n\right),\left(n\right)}=\binom{n}{n}=1$.
Thus, all constants $c_{\kappa,\lambda}$ are obtained, so is $C_{\lambda}(Y)$. 
\end{rem}

\subsection{Functions}

~\\
~
\noindent \begin{flushleft}
\textbf{Calmi}(\emph{partition)}
\par\end{flushleft}

\noindent \begin{flushleft}
calculate the product of factorials in front of the sum on the right-hand
side of \eqref{eq:MZonalComputation}.
\par\end{flushleft}

\noindent \begin{flushleft}
\rule[0.5ex]{1\columnwidth}{1pt}
\par\end{flushleft}

\noindent \begin{flushleft}
\textbf{MZonal}(\emph{partition},\emph{variables)}
\par\end{flushleft}

\noindent \begin{flushleft}
computes the monomial symmetric function using \eqref{eq:MZonalComputation}.
\par\end{flushleft}

\noindent \begin{flushleft}
\rule[0.5ex]{1\columnwidth}{1pt}
\par\end{flushleft}

\noindent \begin{flushleft}
\textbf{SumVariable}\emph{($\kappa$,$\lambda$)}
\par\end{flushleft}

\noindent \begin{flushleft}
computes the list of $\mu$ appearing in the summation of \eqref{eq:CRec},
according to the paragraph under.
\par\end{flushleft}

\noindent \begin{flushleft}
\rule[0.5ex]{1\columnwidth}{1pt}
\par\end{flushleft}

\noindent \begin{flushleft}
\textbf{RHO}(\emph{partition)}
\par\end{flushleft}

\noindent \begin{flushleft}
computes $\rho_{\lambda}$ from \eqref{eq:RHO}.
\par\end{flushleft}

\noindent \begin{flushleft}
\rule[0.5ex]{1\columnwidth}{1pt}
\par\end{flushleft}

\noindent \begin{flushleft}
\textbf{Coeffi}\emph{(kappa,lambda)}
\par\end{flushleft}

\noindent \begin{flushleft}
computes the constants $c_{\kappa,\lambda}$ through \eqref{eq:CRec}
and \eqref{eq:CInitial}.
\par\end{flushleft}

\noindent \begin{flushleft}
\rule[0.5ex]{1\columnwidth}{1pt}
\par\end{flushleft}

\noindent \begin{flushleft}
\textbf{CZonal}\emph{(partition},\emph{variables)}
\par\end{flushleft}

\noindent \begin{flushleft}
computes the zonal polynomial $C_{\lambda}\left(y_{1},\ldots,y_{m}\right)$
through linear combination \eqref{eq:CInTermsOfM}.
\par\end{flushleft}

\subsection{Examples}
\noindent \begin{flushleft}
In{[}1{]}= \textbf{Coeffi({[}2, 1{]}, {[}1, 1, 1{]})}\\
~\\
Out{[}1{]}=$18/5$\\
~\\
In{[}2{]}= \textbf{MZonal({[}2, 1{]}, {[}a, b, c{]})}\\
~\\
Out{[}2{]}=$a^{2}b+ab^{2}+a^{2}c+b^{2}c+ac^{2}+bc^{2}$\\
~\\
In{[}2{]}= \textbf{CZonal{[}{[}2, 1{]}, {[}a, b, c{]})}\\
~\\
Out{[}2{]}=$\frac{12}{5}\left(a^{2}b+ab^{2}+a^{2}c+b^{2}c+ac^{2}+bc^{2}\right)+\frac{18}{5}abc$
\par\end{flushleft}

\includegraphics[scale=0.3]{ScreenShot}


\section{Infinite Families of Coefficients $c_{\kappa,\lambda}$}

In this section, we study some infinite families among the coefficients
$c_{\kappa,\lambda}$ of the zonal polynomials~$C_{\kappa}(Y)$. Since we have
seen that $c_{(n),(n)}=1$ for all~$n$ (that corresponds to the upper left
corner of the $c_{\kappa,\lambda}$-matrix), one can ask whether the
``neighboring'' entries also admit a closed form for general~$n$.  The
following theorem gives an explicit answer for the first few cases.

\begin{thm}
\label{thm:fam1}
We have
\begin{alignat*}{2}
  c_{(n),(n-1,1)} &= \frac{n}{2n-1}, &
  c_{(n),(n-2,2)} &= \frac{3n(n-1)}{2(2n-1)(2n-3)}, \\
  c_{(n-1,1),(n-1,1)} &= \frac{2n(n-1)}{2n-1}, &
  c_{(n-1,1),(n-2,2)} &= \frac{2n(n-1)(n-2)}{(2n-1)(2n-5)}, \\
  &&\qquad
  c_{(n-2,2),(n-2,2)} &= \frac{2n(n-1)(n-2)(n-3)}{(2n-3)(2n-5)},
\end{alignat*}
provided that $n\geq2$ (left column) resp. $n\geq4$ (right column) holds.
\end{thm}
\begin{proof}
Obviously, $\rho_{\left(n\right)}=n\cdot\left(n-1\right)$ and
$\rho_{\left(n-1,1\right)}=\left(n-1\right)\left(n-1-1\right)+1\cdot\left(1-2\right)=\left(n-1\right)n-2n+1$,
so that 
\[
  \rho_{\left(n\right)}-\rho_{\left(n-1,1\right)}=2n-1.
\]
Moreover, the only possible $\mu$ satisfying $\left(n-1,1\right)<\mu\leq\left(n\right)$
is $\mu=\left(n\right)$ with $t=1$. Therefore,  by applying \eqref{eq:CRec}, we obtain
\[
  c_{\left(n\right),\left(n-1,1\right)}=\frac{\left(n-1+1\right)-\left(1-1\right)}{2n-1}\cdot1=\frac{n}{2n-1}.
\]
Using \eqref{eq:CInitial},
\[
  c_{\left(n\right),\left(n-1,1\right)}+c_{\left(n-1,1\right),\left(n-1,1\right)}=\binom{n}{n-1,1}=n,
\]
we get the next coefficient:
\[
  c_{\left(n-1,1\right),\left(n-1,1\right)}=n-\frac{n}{2n-1}=\frac{2n^{2}-2n}{2n-1}=\frac{2n\left(n-1\right)}{2n-1}.
\]
A direct calculation gives 
\[
  \rho_{\left(n\right)}-\rho_{\left(n-2,2\right)} =
  n\left(n-1\right)-\left(n-2\right)\left(n-3\right) =
  2\left(2n-3\right).
\]
Also, if $\left(n-2,2\right)<\mu\leq\left(n\right)$, then either
$\mu=\left(n-1,1\right)$ or $\mu=\left(n\right)$. Thus, the coefficients
$c_{\kappa,(n-2,2)}$ are obtained by applying \eqref{eq:CRec} and~\eqref{eq:CInitial}
in the same, symbolic, fashion:
\begin{align*}
  c_{\left(n\right),\left(n-2,2\right)} 
  &= \frac{\left(n-2+1\right)-\left(2-1\right)}{2\left(2n-3\right)}\cdot\frac{n}{2n-1}+\frac{\left(n-2+2\right)-\left(2-2\right)}{2\left(2n-3\right)}\cdot1\\
  &= \frac{3n\left(n-1\right)}{2\left(2n-1\right)\left(2n-3\right)},\\
  c_{\left(n-1,1\right),\left(n-2,2\right)}
  &= \frac{\left(n-2+1\right)-\left(2-1\right)}{\rho_{\left(n-1,1\right)}-\rho_{\left(n-2,2\right)}}\cdot\frac{2n\left(n-1\right)}{2n-1}\\
  &= \frac{\left(n-2+1\right)-\left(2-1\right)}{\left(n-1\right)n-2n+1-\left(n-2\right)\left(n-3\right)}\cdot\frac{2n\left(n-1\right)}{2n-1}\\
  &= \frac{2n\left(n-1\right)\left(n-2\right)}{\left(2n-1\right)\left(2n-5\right)},\\
  c_{\left(n-2,2\right),\left(n-2,2\right)}
  &= \binom{n}{n-2,2}-c_{\left(n\right),\left(n-2,2\right)}-c_{\left(n-1,1\right),\left(n-2,2\right)}\\
  &= \frac{n\left(n-1\right)}{2}-\frac{3n\left(n-1\right)}{2\left(2n-1\right)\left(2n-3\right)}-\frac{2n\left(n-1\right)\left(n-2\right)}{\left(2n-1\right)\left(2n-5\right)}\\
  &= \frac{2n\left(n-1\right)\left(n-2\right)\left(n-3\right)}{\left(2n-3\right)\left(2n-5\right)}.
\end{align*}
\end{proof}

% TODO: not clear whether we need this.
%% Now, consider $\kappa=\left(n-n',\lambda'\right)$ and $\lambda=\left(n-n'',\lambda''\right)$.
%% To guarantee $\kappa\geq\lambda$, we have either $n'<n''$, or $n'=n''$
%% and $\lambda'\geq\lambda''$. In addition, we assume that 
%% \[
%% \lambda'=\left(\lambda'_{1},\ldots,\lambda'_{k}\right)\ \ \ \text{and}\ \ \ \lambda''=\left(\lambda''_{1},\ldots,\lambda''_{l}\right).
%% \]
%% Recall that 
%% \[
%% \rho_{\lambda'}=\sum_{i=1}^{k}\lambda'_{i}\left(\lambda'_{i}-i\right),
%% \]
%% we see
%% \begin{align*}
%% \rho_{\kappa} & =\left(n-n'\right)\left(n-n'-1\right)+\sum_{i=1}^{k}\lambda'_{i}\left(\lambda'_{i}-i-1\right)\\
%%  & =\left(n-n'\right)^{2}-\left(n-n'\right)+\sum_{i=1}^{k}\lambda'_{i}\left(\lambda'_{i}-i\right)-\sum_{i=1}^{k}\lambda'_{i}\\
%%  & =\left(n-n'\right)^{2}-\left(n-n'\right)+\rho_{\lambda'}-n'\\
%%  & =\left(n-n'\right)^{2}-n+\rho_{\lambda'}.
%% \end{align*}
%% Similarly, 
%% \[
%% \rho_{\lambda}=\left(n-n''\right)^{2}-n+\rho_{\lambda''},
%% \]
%% which implies
%% \[
%% \rho_{\kappa}-\rho_{\lambda}=\rho_{\lambda'}-\rho_{\lambda''}+2n\left(n''-n'\right)+\left(n'^{2}-n''^{2}\right).
%% \]


When $n$ is sufficiently large, the lexicographically largest elements of
$\mathcal{P}_n$ are (in descending order): $(n)$, $(n-1,1)$, $(n-2,2)$,
$(n-2,1,1)$, $(n-3,3)$, etc. In Table~\ref{tab:cn1} we give closed forms for
the coefficients $c_{\kappa,\lambda}$ when $\kappa$ and $\lambda$ are taken
from these lexicographically largest partitions, i.e., when both $\kappa$ and
$\lambda$ are of the form $(n-n',\lambda')$, where $\lambda'$ is a partition
of~$n'$. Pictorially speaking, this table represents the upper left submatrix
of the $c_{\kappa,\lambda}$-matrix for large~$n$; more precisely, for $n\geq6$
since we consider only partitions $(n),\dots,(n-3,1,1,1)$. We remark that all
these formulas can be rigorously proven by applying \eqref{eq:CRec}
and~\eqref{eq:CInitial} for symbolic~$n$, as it was done in the proof of
Theorem~\ref{thm:fam1}.  Since these manipulations are too tedious to be done
by hand, we have implemented the corresponding procedure in our Mathematica
package. % TODO: reference

\begin{table}
\[
  \def\arraystretch{1.3}
  \begin{array}{c|ccccc}
    \kappa\backslash\lambda &
    (n) & (n-1,1) & (n-2,2) & (n-2,1,1) & (n-3,3) \\ \hline
    \rule{0pt}{14pt} (n) &
    1 & \frac{n}{2 n-1} & \frac{3 (n-1) n}{2 (2 n-3) (2 n-1)} & \frac{(n-1)
      n}{(2 n-3) (2 n-1)} & \frac{5 (n-2) (n-1) n}{2 (2 n-5) (2 n-3) (2 n-1)}
    \\[1ex]
    (n-1,1) &
    0 & \frac{2 (n-1) n}{2 n-1} & \frac{2 (n-2) (n-1) n}{(2 n-5) (2 n-1)} &
    \frac{2 n (2 n^2-6 n+3)}{(2 n-5) (2 n-1)} & \frac{3 (n-3) (n-2)
      (n-1) n}{(2 n-7) (2 n-5) (2 n-1)} \\[1ex]
    (n-2,2) &
    0 & 0 & \frac{2 (n-3) (n-2) (n-1) n}{(2 n-5) (2 n-3)} & \frac{4 (n-3)
      (n-2) (n-1) n}{3 (2 n-5) (2 n-3)} & \frac{2 (n-4) (n-3) (n-2) (n-1)
      n}{(2 n-9) (2 n-5) (2 n-3)} \\[1ex]
    (n-2,1,1) &
    0 & 0 & 0 & \frac{2}{3} (n-2) n & 0 \\[1ex]
    (n-3,3) &
    0 & 0 & 0 & 0 & \frac{4 (n-5) (n-4) (n-3) (n-2) (n-1) n}{3 (2 n-9) (2 n-7) (2 n-5)}
  \end{array}
\]
\[
  \def\arraystretch{1.3}
  \begin{array}{c|ccccc}
    \kappa\backslash\lambda &
    (n-3,2,1) & (n-3,1,1,1) % & (n-4,4) & (n-4,3,1) & (n-4,2,2)
    \\ \hline
    \rule{0pt}{14pt} (n) &
    \frac{3 (n-2) (n-1) n}{2 (2 n-5) (2 n-3) (2 n-1)}
    & \frac{(n-2) (n-1) n}{(2 n-5) (2 n-3) (2 n-1)} 
    % & \frac{35 (n-3) (n-2) (n-1) n}{8 (2 n-7) (2 n-5) (2 n-3) (2 n-1)}
    % & \frac{5 (n-3) (n-2) (n-1) n}{2 (2 n-7) (2 n-5) (2 n-3) (2 n-1)}
    % & \frac{9 (n-3) (n-2) (n-1) n}{4 (2 n-7) (2 n-5) (2 n-3) (2 n-1)}
    \\[1ex]
    (n-1,1) &
    \frac{(n-2) n (5 n^2-20 n+11)}{(2 n-7) (2 n-5) (2 n-1)}
    & \frac{6 (n-2) n (n^2-4 n+2)}{(2 n-7) (2 n-5) (2 n-1)}
    % & \frac{5 (n-4) (n-3) (n-2) (n-1) n}{(2 n-9) (2 n-7) (2 n-5) (2 n-1)}
    % & \frac{(n-3) (n-2) n (8 n^2-40 n+23)}{(2 n-9) (2 n-7) (2 n-5) (2 n-1)}
    % & \frac{6 (n-3) (n-2) n (n^2-5 n+3)}{(2 n-9) (2 n-7) (2 n-5) (2 n-1)}
    \\[1ex]
    (n-2,2) &
    \frac{2 (n-3) (n-1) n (5 n^2-30 n+36)}{3 (2 n-9) (2 n-5) (2 n-3)}
    & \frac{4 (n-3) (n-1) n (n^2-6 n+7)}{(2 n-9) (2 n-5) (2 n-3)}
    % & \frac{3 (n-5) (n-4) (n-3) (n-2) (n-1) n}{(2 n-11) (2 n-9) (2 n-5) (2 n-3)}
    % & \frac{4 (n-4) (n-3) (n-1) n (n^2-7 n+9)}{(2 n-11) (2 n-9) (2 n-5) (2 n-3)}
    % & \frac{2 (n-1) n (11 n^4-154 n^3+761 n^2-1554 n+1098)}{3 (2 n-11) (2 n-9) (2 n-5) (2 n-3)}
    \\[1ex]
    (n-2,1,1) &
    \frac{2 (n-3) (n-2) n}{3 (2 n-7)}
    & \frac{2 (n-2) n (2 n^2-9 n+8)}{(2 n-7) (2 n-3)}
    % & 0
    % & \frac{(n-4) (n-3) (n-2) n}{(2 n-9) (2 n-7)}
    % & \frac{2 (n-4) (n-3) (n-2) n}{3 (2 n-9) (2 n-7)}
    \\[1ex]
    (n-3,3) &
    \frac{4 (n-5) (n-4) (n-3) (n-2) (n-1) n}{5 (2 n-9) (2 n-7) (2 n-5)}
    & \frac{8 (n-5) (n-4) (n-3) (n-2) (n-1) n}{15 (2 n-9) (2 n-7) (2 n-5)}
    % & \frac{4 (n-6) (n-5) (n-4) (n-3) (n-2) (n-1) n}{3 (2 n-13) (2 n-9) (2 n-7) (2 n-5)}
    % & \frac{4 (n-5) (n-4) (n-2) (n-1) n (8 n^2-72 n+135)}{15 (2 n-13) (2 n-9) (2 n-7) (2 n-5)}
    % & \frac{8 (n-5) (n-4) (n-2) (n-1) n (n^2-9 n+17)}{5 (2 n-13) (2 n-9) (2 n-7) (2 n-5)}
    \\[1ex]
    (n-3,2,1) &
    \frac{4 (n-4) (n-3) (n-1) n}{5 (2 n-7)}
    & \frac{6 (n-4) (n-3) (n-1) n}{5 (2 n-7)}
    % & 0
    % & \frac{4 (n-5) (n-4) (n-3) (n-1) n}{5 (2 n-11) (2 n-7)}
    % & \frac{4 (n-3) (n-1) n (2 n^2-18 n+39)}{5 (2 n-11) (2 n-7)}
    \\[1ex]
    (n-3,1,1,1) &
    0 & \frac{2 (n-3) (n-2) (n-1) n}{3 (2 n-3)}
    % & 0 & 0 & 0
    \\[1ex]
    (n-4,4) &
    0 & 0
    % & \frac{2 (n-7) (n-6) (n-5) (n-4) (n-3) (n-2) (n-1) n}{3 (2 n-13) (2 n-11) (2 n-9) (2 n-7)}
    % & \frac{8 (n-7) (n-6) (n-5) (n-4) (n-3) (n-2) (n-1) n}{21 (2 n-13) (2 n-11) (2 n-9) (2 n-7)}
    % & \frac{12 (n-7) (n-6) (n-5) (n-4) (n-3) (n-2) (n-1) n}{35 (2 n-13) (2 n-11) (2 n-9) (2 n-7)}
    %% \\[1ex]
    %% (n-4,3,1) &
    %% 0 & 0
    %% % & 0
    %% % & \frac{4 (n-6) (n-5) (n-4) (n-2) (n-1) n}{7 (2 n-11) (2 n-9)}
    %% % & \frac{8 (n-6) (n-5) (n-4) (n-2) (n-1) n}{21 (2 n-11) (2 n-9)}
    %% \\[1ex]
    %% (n-4,2,2) &
    %% 0 & 0
    %% % & 0 & 0 & \frac{2}{15} (n-5) (n-2) (n-1) n
  \end{array}
\]
\caption{Coefficients $c_{\kappa,\lambda}$ for some of the lexicographically
  largest partitions of~$n$; the lower table continues the upper one to the
  right.}
\label{tab:cn1}
\end{table}
\begin{example}
For example, for $n=23$ we obtain the coefficient $c_{(21,2),(21,1,1)}$ by
looking at the entry in Table~\ref{tab:cn1} in row $(n-2,2)$ and column
$(n-2,1,1)$:
\[
  c_{(21,2),(21,1,1)} = \frac{4 (n-3) (n-2) (n-1) n}{3 (2 n-5) (2 n-3)} \Big|_{n\to23}
  = \frac{4 \cdot 20 \cdot 21 \cdot 22 \cdot 23}{3 \cdot 41 \cdot 43}
  = \frac{283360}{1763}.
\]
\end{example}

The next families we study are located in the ``lower right corner'' of
$c_{\kappa,\lambda}$-matrix. In this case both $\kappa$ and $\lambda$ are of
the form $(2^m,1^{n-2m})$, i.e., a sequence of $m$ times the part~$2$ and
$n-2m$ times the part~$1$. Note that for $m=0,1,2,\dots$ we obtain the
lexicographically smallest partitions of~$n$ (in increasing order).  In
contrast to Theorem~\ref{thm:fam1} it is not straightforward to derive
formulas for general~$n$ by means of Equations \eqref{eq:CRec}
and~\eqref{eq:CInitial}. Instead, we compute the first few values of these
sequences, say up to $n=30$, and then guess a closed form expression. The
first few results are displayed in Table~\ref{tab:cn2}.

More precisely, we first applied the \texttt{Guess} package~\cite{Guess} to
find a plausible candidate for a linear recurrence with polynomial
coefficients (in all considered instances this recurrence was of first order,
which easily allowed for a closed form solution). Then, after observing that
all expressions obtained this way were of the form $2^n\cdot r(n)$ where
$r(n)$ is some rational function in~$n$, we refined our ansatz to only search
for expressions of this form: divide the $n$-th sequence entry by $2^n$ and
then perform polynomial interpolation and rational reconstruction.

\begin{table}
\[
  \def\arraystretch{1.3}
  \begin{array}{c|ccc}
    \kappa\backslash\lambda
    % & (2^6,1^{n-12}) & (2^5,1^{n-10})
    & (2^4,1^{n-8}) & (2^3,1^{n-6}) & (2^2,1^{n-4})
    % & (2,1^{n-2}) & (1^n)
    \\ \hline
    %% (2^6,1^{n-12})
    %% & \frac{1}{315} 2^{n-4} (n-10) (n-9) (n-2) (n-1) n
    %% & \frac{1}{315} 2^{n-4} (n-11) (n-10)^2 (n-2) (n-1) n
    %% & \frac{1}{315} 2^{n-5} (n-11) (n-10)^2 (n-9) (n-2) (n-1) n
    %% & \frac{1}{945} 2^{n-5} (n-11) (n-10)^2 (n-9) (n-6) (n-2) (n-1) n
    %% & \frac{1}{945} 2^{n-7} (n-11) (n-10)^2 (n-9) (n-5) (n-4) (n-2) (n-1) n
    %% & ?
    %% & ?
    %% \\[1ex]
    %% (2^5,1^{n-10})
    %% & 0
    %% & \frac{1}{45} 2^{n-4} (n-8) (n-7) (n-1) n
    %% & \frac{1}{45} 2^{n-4} (n-9) (n-8)^2 (n-1) n
    %% & \frac{1}{45} 2^{n-5} (n-9) (n-8)^2 (n-7) (n-1) n
    %% & \frac{1}{135} 2^{n-5} (n-9) (n-8)^2 (n-7) (n-4) (n-1) n
    %% & \frac{1}{135} 2^{n-7} (n-9) (n-8)^2 (n-7) (n-3) (n-2) (n-1) n
    %% & \frac{1}{675} 2^{n-7} (n-9) (n-8)^2 (n-7) (n-2) (n-1)^2 n^2
    %% \\[1ex]
    \rule{0pt}{14pt}
    (2^4,1^{n-8})
    % & 0 & 0
    & \frac{2^{n-3} (n-6) (n-5) n}{15}
    & \frac{2^{n-3} (n-7) (n-6)^2 n}{15}
    & \frac{2^{n-4} (n-7) (n-6)^2 (n-5) n}{15}
    % & \frac{2^{n-4} (n-7) (n-6)^2 (n-5) (n-2) n}{45}
    % & \frac{2^{n-6} (n-7) (n-6)^2 (n-5) (n-1) n^2}{45}
    \\[1ex]
    (2^3,1^{n-6})
    % & 0 & 0
    & 0
    & \frac{2^{n-3} (n-4) (n-3)}{3}
    & \frac{2^{n-3} (n-5) (n-4)^2}{3}
    % & \frac{2^{n-4} (n-5) (n-4)^2 (n-3)}{3}
    % & \frac{2^{n-4} (n-5) (n-4)^2 (n-3) n}{9}
    \\[1ex]
    (2^2,1^{n-4})
    % & 0 & 0
    & 0 & 0
    & \frac{2^{n-1} (n-2) (n-1)}{3 (n+1)}
    % & \frac{2^{n-1} (n-3) (n-2)^2}{3 (n+1)}
    % & \frac{2^{n-2} (n-3) (n-2)^2 (n-1)}{3 (n+1)}
    \\[1ex]
    (2,1^{n-2})
    % & 0 & 0
    & 0 & 0 & 0
    % & \frac{2^{n-1} n}{n+2}
    % & \frac{2^{n-1} (n-1) n^2}{(n+1) (n+2)}
    \\[1ex]
    (1^n)
    % & 0 & 0
    & 0 & 0 & 0
    % & 0 & \frac{2^n}{n+1}
\end{array}
\]
\[
  \def\arraystretch{1.3}
  \begin{array}{c|cc}
    \kappa\backslash\lambda
    & (2,1^{n-2}) & (1^n) \\ \hline
    \rule{0pt}{14pt}
    (2^4,1^{n-8})
    & \frac{2^{n-4} (n-7) (n-6)^2 (n-5) (n-2) n}{45}
    & \frac{2^{n-6} (n-7) (n-6)^2 (n-5) (n-1) n^2}{45}
    \\[1ex]
    (2^3,1^{n-6})
    & \frac{2^{n-4} (n-5) (n-4)^2 (n-3)}{3}
    & \frac{2^{n-4} (n-5) (n-4)^2 (n-3) n}{9}
    \\[1ex]
    (2^2,1^{n-4})
    & \frac{2^{n-1} (n-3) (n-2)^2}{3 (n+1)}
    & \frac{2^{n-2} (n-3) (n-2)^2 (n-1)}{3 (n+1)}
    \\[1ex]
    (2,1^{n-2})
    & \frac{2^{n-1} n}{n+2}
    & \frac{2^{n-1} (n-1) n^2}{(n+1) (n+2)}
    \\[1ex]
    (1^n)
    & 0 & \frac{2^n}{n+1}
\end{array}
\]
\caption{Coefficients $c_{\kappa,\lambda}$ for some of the lexicographically
  smallest partitions of~$n$; the lower table continues the upper one to the
  right.}
\label{tab:cn2}
\end{table}

Muirhead~\cite[Lemma~7.2.3]{Muirhead} gives a necessary condition for some
coefficients $c_{\kappa,\lambda}$ to be zero, but without proof. We recall his
result here and give a simple proof of it.
\begin{lemma}\label{lem:zero1}
  Let $n\in\mathbb{N}$ and $\kappa,\lambda$ be partitions of~$n$ with $p$ and
  $q$ parts, respectively. If $p>q$, i.e., if $\kappa$ has more parts
  than~$\lambda$, then $c_{\kappa,\lambda}=0$.
\end{lemma}
\begin{proof}
  From the definition of zonal polynomials it follows
  \begin{align*}
    0 = C_{\kappa}(y_1,\dots,y_{p-1}) &=
    \sum_{\lambda\leq\kappa} c_{\kappa,\lambda} M_{\lambda}(y_1,\dots,y_{p-1}) \\
    &= \!\!\!\!\!\sum_{\genfrac{}{}{0pt}{}{\lambda\leq\kappa}{\operatorname{parts}(\lambda)<p}}\!\!\!\!\!
      c_{\kappa,\lambda} \underbrace{M_{\lambda}(y_1,\dots,y_{p-1})}_{\neq0} \quad +
    \sum_{\genfrac{}{}{0pt}{}{\lambda\leq\kappa}{\operatorname{parts}(\lambda)\geq p}}\!\!\!\!\!
      c_{\kappa,\lambda} \underbrace{M_{\lambda}(y_1,\dots,y_{p-1})}_{=0}
  \end{align*}
  from which it becomes apparent that all coefficients $c_{\kappa,\lambda}$
  for which $\lambda$ has fewer parts than $p=\operatorname{parts}(\kappa)$
  must be zero.
\end{proof}

While Lemma~\ref{lem:zero1} only gives a necessary condition under which
$c_{\kappa,\lambda}$ is zero, we want to obtain a full characterization, i.e.,
a necessary and sufficient condition for $c_{\kappa,\lambda}=0$. For example,
we have $c_{(8,2,2),(7,4,1)} = 0$ although both partitions have the same length.
\begin{lemma}\label{lem:zero2}
  Let $n\in\mathbb{N}$ and let $\kappa$ and $\lambda$ be two partitions of~$n$
  with $\kappa\geq\lambda$ in lexicographic order. Then $c_{\kappa,\lambda}=0$
  if and only if there exists $j\in\mathbb{N}$ such that
  \[
    \sum_{i=1}^j (\kappa_i - \lambda_i) < 0,
  \]
  where the partitions are filled with zeros as necessary.
\end{lemma}
\begin{proof}
  TODO
\end{proof}

\begin{example}
  For $\kappa=(8,2,2)$ and $\lambda=(7,4,1)$ we verify that the second
  partial sum of their differences, i.e. $j=2$, is negative:
  $(8-7)+(2-4)=-1$, and hence $c_{\kappa,\lambda}$ must be zero.
\end{example}


We now present a closed form of the coefficients $c_{\kappa,\lambda}$ when
both $\kappa$ and $\lambda$ have at most two parts.
\begin{thm}
  If $n=2m$ is an even integer with $n\geq2$, then for $0\leq j\leq i\leq m$
  we have
  \begin{equation}\label{eq:cf2even}
    %% c_{(n/2+i,n/2-i),(n/2+j,n/2-j)} = \alpha_{i,j}\cdot
    %% \frac{2^n \, (n/2 - i + 1)_i}{((n+1)/2)_{i+1}}
    c_{(m+i,m-i),(m+j,m-j)} = \alpha_{i,j}\cdot 2^{2m}\cdot
    \frac{\bigl(m-i+1\bigr)_{\!i}}{\bigl(m+\frac12\bigr)_{\!i+1}}
  \end{equation}
  where $\alpha_{0,0}=\frac12$ and
  \[
    \alpha_{i,j} = \frac{(-1)^{i+j} \, (4i+1) \, \bigl(i-j+1\bigr)_{\!j-1} \,
      \bigl(\frac12-i+j\bigr)_{\!2i}}{2 \, (i-1)! \, (i+j)!}
    \qquad (i\geq1,\,0\leq j\leq i).
  \]
  If $n=2m+1$ is an odd integer with $n\geq3$, then for $0\leq j\leq i\leq m$
  we have
  \begin{equation}\label{eq:cf2odd}
    c_{(m+1+i,m-i),(m+1+j,m-j)} = \beta_{i,j}\cdot 2^{2m+1}\cdot
    \frac{\bigl(m-i+1\bigr)_{\!i}}{\bigl(m+\frac32\bigr)_{\!i+1}}
  \end{equation}
  where $\beta_{0,0}=\frac34$ and
  \[
    \beta_{i,j} = \frac{(-1)^{i+j} \, (4i+3) \, \bigl(i-j+1\bigr)_{\!j-1} \,
      \bigl(\frac12-i+j\bigr)_{\!2i+1}}{2 \, (i-1)! \, (i+j+1)!}
    \qquad (i\geq1,\,0\leq j\leq i).
  \]
  In the case $i=m$ the notation $(m+i,m-i)$ is understood to denote the
  partition~$(2m)$.
\end{thm}
\begin{proof}
  The proof is based on the recursions \eqref{eq:CRec}
  and~\eqref{eq:CInitial}.  We are interested in the coefficients
  $c_{\kappa,\lambda}$ where both partitions $\kappa$ and $\lambda$ have at
  most two parts. Then, on the one hand, the partitions~$\mu$ that appear in
  the recursive formula~\eqref{eq:CRec} have also at most two parts, by the
  way how they are constructed from~$\lambda$. On the other hand, when we use
  Equation~\eqref{eq:CInitial} to compute $c_{\lambda,\lambda}$, only those
  coefficients $c_{\kappa,\lambda}$ contribute for which $\kappa$ has at most
  two parts; this is a direct consequence of Lemma~\ref{lem:zero1}.  We
  conclude that for the computation of $c_{\kappa,\lambda}$ we do not need any
  $c_{\kappa',\lambda'}$ with $\kappa'$ or $\lambda'$ having more than two
  parts.

  Specializing Equation~\eqref{eq:CRec} to partitions with at most two parts,
  we obtain the following recursion:
  \begin{equation}\label{eq:CRec2}
    c_{(m+i,m-i),(m+j,m-j)} = \sum_{k=0}^{i-j-1}
    \frac{(i - k) c_{(m+i,m-i),(m+i-k,m-i+k)}}{(i - j)\bigl(i + j + \frac12\bigr)}.
  \end{equation}
  Similarly, Equation~\eqref{eq:CInitial} turns into
  \begin{equation}\label{eq:CInitial2}
    \sum_{k=i}^m c_{(m+k,m-k),(m+i,m-i)} = \binom{2m}{m+i}.
  \end{equation}
  In the rest of the proof we will show that the asserted closed-form
  expression is the unique solution to the recursive equations
  \eqref{eq:CRec2} and~\eqref{eq:CInitial2}.

  Equation~\eqref{eq:CRec2} is used to calculate $c_{(m+i,m-i),(m+j,m-j)}$
  when $i>j$, the case $i=j$ is covered by~\eqref{eq:CInitial}. After
  plugging the closed form~\eqref{eq:cf2even} into \eqref{eq:CRec2}, the
  right-hand side turns into
  \begin{equation}\label{eq:rhs}
    \frac{2^{2m} (4i+1) \bigl(m-i+1\bigr)_{\!i}}{(i-1)! \, (i-j) (2i+2j+1) \bigl(m+\frac12\bigr)_{\!i+1}}
    \sum_{k=0}^{i-j-1} \frac{(-1)^k (i-k) \bigl(\frac12-k\bigr)_{\!2i} \bigl(k+1\bigr)_{\!i-k-1}}{(2i-k)!}.
  \end{equation}
  Using special-purpose computer algebra packages, such as the
  HolonomicFunctions package~\cite{HolonomicFunctions}, we find that the
  expression in the sum, denote it by $f(k,i)$, is Gosper-summable. More
  precisely, we find a function
  \[
    g(k,i) = \frac{(4i-2k+1) \, k}{2 \, (i-k)} \cdot f(k,i) =
    \frac{(-1)^k k \, (4i-2k+1) \, \bigl(\frac12-k\bigr)_{\!2i} \bigl(k+1\bigr)_{\!i-k-1}}{2 \, (2i-k)!}
  \]
  with the property $g(k+1,i)-g(k,i)=f(k,i)$ (the latter can be easily
  verified). By telescoping, and by noting that $g(0,i)=0$, we obtain
  the value of the sum in~\eqref{eq:rhs}:
  \[
    g(i-j,i) = \frac{(-1)^{i-j} (i-j) \, (2i+2j+1) \, \bigl(i-j+1\bigr)_{\!j-1}
      \bigl(\frac12-i+j\bigr)_{\!2i}}{2 \, (i+j)!}.
  \]
  Hence a closed form for \eqref{eq:rhs} is given by
  \[
    \frac{(-1)^{i-j} \, 2^{2m-1} (4i+1) \, \bigl(m-i+1\bigr)_{\!i} \, \bigl(i-j+1\bigr)_{\!j-1}
      \, \bigl(\frac12-i+j\bigr)_{\!2i}}{(i-1)! \, (i+j)! \, \bigl(m+\frac12\bigr)_{\!i+1}}
  \]
  which is exactly the asserted closed form of $c_{(m+i,m-i),(m+j,m-j)}$.
  Thus \eqref{eq:cf2even} satisfies \eqref{eq:CRec2}; it remains to show that
  it also satisfies~\eqref{eq:CInitial2}. TODO...

  TODO: odd case \eqref{eq:cf2odd} is analogous.
\end{proof}


\section{Partitions with two parts}
We concentrate on the coefficients $c_{\kappa,\lambda}$ with $\lambda=\left(n',m'\right)\leq\left(n,m\right)=\kappa$. 
\begin{enumerate}
\item First of all, it is easy to compute that 
\[
\rho_{\kappa}=n\left(n-1\right)+m\left(m-2\right)\ \ \ \text{and}\ \ \ \rho_{\lambda}=n'\left(n'-1\right)+m'\left(m'-2\right),
\]
so that, noting $N=n+m=n'+m'$, 
\[
\rho_{\kappa}-\rho_{\lambda}=n^{2}+m^{2}-m-n'^{2}-m'^{2}+m'.
\]
Or, equivalently, by letting $d=n-n'$, we have 
\begin{align*}
\rho_{\kappa}-\rho_{\lambda} & =n^{2}+m^{2}-m-\left(n-d\right)^{2}-\left(m+d\right)^{2}+m+d\\
 & =2d\left(n-m\right)-2d^{2}+d.
\end{align*}
For simplicity, let $s=n-m$, so that $n'-m'=s-2d$ and $\rho_{\kappa}-\rho_{\lambda}=2ds-2d^{2}+d=d\left(2s-2d+1\right)$. 
\item By \textbackslash{}eqref\{eq:CRec\}, noting that $\lambda=\left(n',m'\right)$
only has two parts,
\[
c_{\kappa,\lambda}=\sum_{\lambda<\mu\leq\kappa}\frac{\left(n'+t\right)-\left(m'-t\right)}{\rho_{\kappa}-\rho_{\lambda}}c_{\kappa,\mu}=\sum_{t=1}^{d}\frac{s-2d+2t}{d\left(2s-2d+1\right)}c_{\kappa,\mu}.
\]
\item We begin with some concrete examples. For simplicity, let $M=n-m$. 
\begin{itemize}
\item For $\lambda=\left(n-1,m+1\right)$, 
\[
c_{\left(n,m\right),\left(n-1,m-1\right)}=\frac{s}{2s-1}c_{\left(n,m\right),\left(n,m\right)}.
\]
\item For $\lambda=\left(n-2,m+2\right)$, 
\begin{align*}
c_{\left(n,m\right),\left(n-2,m-2\right)} & =\frac{1}{4s-6}\left[\left(s-2\right)c_{\left(n,m\right),\left(n-1,m-1\right)}+sc_{\left(n,m\right),\left(n,m\right)}\right]\\
 & =\frac{c_{\left(n,m\right),\left(n,m\right)}}{4s-6}\left[\frac{\left(s-2\right)s}{2s-1}+s\right]\\
 & =\frac{3s\left(s-1\right)}{2\left(2s-1\right)\left(2s-3\right)}c_{\left(n,m\right),\left(n,m\right)}.
\end{align*}
\item For $\lambda=\left(n-3,m+3\right)$, 
\begin{align*}
c_{\left(n,m\right),\left(n-3,m+3\right)} & =\frac{1}{3\left(2s-5\right)}\left[\left(s-4\right)c_{\left(n,m\right),\left(n-2,m-2\right)}+\left(s-2\right)c_{\left(n,m\right),\left(n-1,m-1\right)}+sc_{\left(n,m\right),\left(n,m\right)}\right]\\
 & =\frac{c_{\left(n,m\right),\left(n,m\right)}}{3\left(2s-5\right)}\left[\left(s-4\right)\frac{3s\left(s-1\right)}{2\left(2s-1\right)\left(2s-3\right)}+\frac{3s\left(s-1\right)}{2s-1}\right]\\
 & =\frac{5s\left(s-1\right)\left(s-2\right)}{2\left(2s-1\right)\left(2s-3\right)\left(2s-5\right)}c_{\left(n,m\right),\left(n,m\right)}.
\end{align*}
\item For $\lambda=\left(n-4,m+4\right)$, 
\begin{align*}
c_{\left(n,m\right),\left(n-4,m+4\right)} & =\frac{1}{4\left(2s-7\right)}\left[\left(s-6\right)c_{\left(n,m\right),\left(n-3,m+3\right)}+c_{\left(n,m\right),\left(n-3,m+3\right)}\cdot3\left(2s-5\right)\right]\\
 & =\frac{7\left(s-3\right)}{4\left(2s-7\right)}c_{\left(n,m\right),\left(n-3,m+3\right)}\\
 & =\frac{7\cdot5s\left(s-1\right)\left(s-2\right)\left(s-3\right)}{4\cdot2\left(2s-1\right)\left(2s-3\right)\left(2s-5\right)\left(2s-7\right)}c_{\left(n,m\right),\left(n,m\right)}\\
 & =\frac{7\cdot5\cdot3s\left(s-1\right)\left(s-2\right)\left(s-3\right)}{4\cdot3\cdot2\left(2s-1\right)\left(2s-3\right)\left(2s-5\right)\left(2s-7\right)}c_{\left(n,m\right),\left(n,m\right)}
\end{align*}
\end{itemize}
\end{enumerate}
\begin{conjecture}
Given two partitions $\kappa=\left(n,m\right)$ and $\lambda=\left(n-d,m+d\right)$
of positive integer $N$, by denoting $s=n-m\geq0$, we have 
\begin{equation}
c_{\kappa,\lambda}=c_{\left(n,m\right),\left(n-d,m+d\right)}=\frac{\left(2d-1\right)!!s\left(s-1\right)\cdots\left(s-d+1\right)}{d!\left(2s-1\right)\left(2s-3\right)\cdots\left(2s-2d+1\right)}c_{\left(n,m\right),\left(n,m\right)},\label{eq:2PartRec}
\end{equation}
where for an odd positive integer $a$, $a!!=a\left(a-2\right)\left(a-4\right)\cdots1$. 
\end{conjecture}

\begin{proof}
By induction. When $d=1$, (\ref{eq:2PartRec}) holds by the computation
above. Suppose (\ref{eq:2PartRec}) holds for $d$. Then, 
\begin{align*}
 & c_{\left(n,m\right),\left(n-d-1,m+d+1\right)}\\
 & =\sum_{t=1}^{d+1}\frac{s-2\left(d+1\right)+2t}{\left(d+1\right)\left(2s-2d-1\right)}c_{\left(n,m\right),\left(n-d-1+t,m+d+1-t\right)}\\
 & =\frac{1}{\left(d+1\right)\left(2s-2d-1\right)}\sum_{t=1}^{d+1}\left(s-2\left(d+1\right)+2t\right)c_{\left(n,m\right),\left(n-d-1+t,m+d+1-t\right)}\\
 & =\frac{1}{\left(d+1\right)\left(2s-2d-1\right)}\left[\left(s-2d\right)c_{\left(n,m\right),\left(n-d,m+d\right)}+\sum_{t=2}^{d+1}\left(s-2\left(d+1\right)+2t\right)c_{\left(n,m\right),\left(n-d-1+t,m+d+1-t\right)}\right]\\
 & =\frac{1}{\left(d+1\right)\left(2s-2d-1\right)}\left[\left(s-2d\right)c_{\left(n,m\right),\left(n-d,m+d\right)}+\sum_{t'=1}^{d}\left(s-2d+t'\right)c_{\left(n,m\right),\left(n-d+t',m+d-t'\right)}\right]\\
 & =\frac{1}{\left(d+1\right)\left(2s-2d-1\right)}\left[\left(s-2d\right)c_{\left(n,m\right),\left(n-d,m+d\right)}+d\left(2s-2d+1\right)c_{\left(n,m\right),\left(n-d,m+d\right)}\right]\\
 & =\frac{\left(2d+1\right)\left(s-d\right)}{\left(d+1\right)\left(2s-2d-1\right)}c_{\left(n,m\right),\left(n-d,m+d\right)}\\
 & =\frac{\left(2d+1\right)\left(s-d\right)}{\left(d+1\right)\left(2s-2d-1\right)}\cdot\frac{\left(2d-1\right)!!s\left(s-1\right)\cdots\left(s-d+1\right)}{d!\left(2s-1\right)\left(2s-3\right)\cdots\left(2s-2d+1\right)}c_{\left(n,m\right),\left(n,m\right)}\\
 & =\frac{\left(2\left(d+1\right)-1\right)!!s\left(s-1\right)\cdots\left(s-\left(d+1\right)+1\right)}{\left(d+1\right)!\left(2s-1\right)\left(2s-3\right)\cdots\left(2s-2\left(d+1\right)+1\right)}c_{\left(n,m\right),\left(n,m\right)}.
\end{align*}
\end{proof}
Now, from \cite[eq.~18, p.~237]{Muirhead} we see when $y_{1}=y_{2}=1$,
\[
\mathcal{C}_{\kappa}\left(1,1\right)=2^{2N}N!\left(2\right)_{n}\left(0\right)_{m}\frac{2\left(n-m\right)-1+2}{\left(2n+1\right)!\left(2m\right)!}=2^{2n}\frac{N!\left(n+1\right)!\left(m-1\right)!\left(2s+1\right)}{\left(2n+1\right)!\left(2m\right)!}.
\]
\begin{itemize}
\item When $n$ and $m$ have different parities, i.e., for all $\lambda=\left(n-d,m+d\right)\leq\kappa$,
$M_{\lambda}\left(1,1\right)=2$. Therefore, we have 
\[
2^{2n}\frac{N!\left(n+1\right)!\left(m-1\right)!\left(2s+1\right)}{\left(2n+1\right)!\left(2m\right)!}=2c_{\left(n,m\right),\left(n,m\right)}\sum_{d=1}^{\frac{s-1}{2}}\frac{\left(2d-1\right)!!s\left(s-1\right)\cdots\left(s-d+1\right)}{d!\left(2s-1\right)\left(2s-3\right)\cdots\left(2s-2d+1\right)},
\]
i.e, 
\[
c_{\left(n,m\right),\left(n,m\right)}=\frac{\left(2n+1\right)!\left(2m\right)!}{2^{2n-1}N!\left(n+1\right)!\left(m-1\right)!\left(2s+1\right)}\sum_{d=1}^{\frac{s-1}{2}}\frac{\left(2d-1\right)!!s\left(s-1\right)\cdots\left(s-d+1\right)}{d!\left(2s-1\right)\left(2s-3\right)\cdots\left(2s-2d+1\right)}.
\]
\item When $n$ and $m$ have the same parity, i.e., $s=n-m$ is even, for
all $\lambda=\left(n-d,m+d\right)\leq\kappa$, $M_{\lambda}\left(1,1\right)=2$,
except that $M_{\left(n-\frac{s}{2},m+\frac{s}{2}\right)}\left(1,1\right)=1$.
Thus, we have 
\[
2^{2n}\frac{N!\left(n+1\right)!\left(m-1\right)!\left(2s+1\right)}{\left(2n+1\right)!\left(2m\right)!}=c_{\left(n,m\right),\left(n,m\right)}\left[2\sum_{d=1}^{\frac{s}{2}-1}\frac{\left(2d-1\right)!!s\left(s-1\right)\cdots\left(s-d+1\right)}{d!\left(2s-1\right)\left(2s-3\right)\cdots\left(2s-2d+1\right)}+\frac{\left(s-1\right)!!s\left(s-1\right)\cdots\left(\frac{s}{2}+1\right)}{\left(\frac{s}{2}\right)!\left(2s-1\right)\left(2s-3\right)\cdots\left(s+1\right)}\right],
\]
which also gives $c_{(n,m),(n,m)}$. 


\begin{thebibliography}{1}
\bibitem{1F1}H.~Hashiguchi, Y.~Numata, N.~Takayama, and A.~Takemura.
The holonomic gradient method for the distribution function of the
largest root of a Wishart matrix. \emph{J.~Multivariate Anal.}~\textbf{117}
(2013), 296\textendash 312.

\bibitem{Helgason}S.~Helgason, \emph{Differential Geometry and Symmetric
Spaces}, Academic Press, 1962.

\bibitem{Representation}R.~Guti\'{e}rrez J\'{a}imez and J.~A.~Mermoso
Guti\'{e}rrez, An application of zonal polynomials to the generalization
of probability distributions, \emph{Linear Algebra Appl.}~\textbf{121}
(1989), 610\textendash 616.

\bibitem{James1}A.~James, Calculation of zonal polynomial coefficients
by use of the Laplace-Beltrami operator, \emph{The Annals of Mathematical
Statistics} \textbf{39} (1968), 1711\textendash 1718.

\bibitem{Proposal}C.~Koutschan, \emph{Algebraic Statistics and Symbolic
Computation}, Project Proposal. 

\bibitem{Moakher}M.~Moakher, A differential geometric approach to
the geometric mean of symmetric positive-definite matrices, \emph{Siam
J.~Matrix Anal.~Appl.}~\textbf{26} (2005), 735\textendash 747.

\bibitem{Muirhead}R.~Muirhead, \emph{Aspects of Multivariate Statistical
Theory}. John Wiley \& Sons Inc., 1982.

\bibitem{Takemura}A.~Takemura, \emph{Zonal Polynomials}, Inst.~Math.~Stat.~Monogr.,
1984.
\end{thebibliography}

\end{document}
